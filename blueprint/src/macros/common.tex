% In this file you should put all LaTeX macros and settings to be used both by
% the pdf version and the web version.
% This should be most of your macros.

% The theorem-like environments defined below are those that appear by default
% in the dependency graph. See the README of leanblueprint if you need help to
% customize this.
% The configuration below use the theorem counter for all those environments
% (this is what the [theorem] arguments mean) and never resets it.
% If you want for instance to number them within chapters then you can add
% [chapter] at the end of the next line.
\newtheorem{theorem}{Theorem}[chapter]
\newtheorem{proposition}[theorem]{Proposition}
\newtheorem{lemma}[theorem]{Lemma}
\newtheorem{corollary}[theorem]{Corollary}

\theoremstyle{definition}
\newtheorem{definition}[theorem]{Definition}
\newtheorem{remark}[theorem]{Remark}
\newtheorem{example}[theorem]{Example}

\newcommand{\proet}[1]{\ensuremath{X_{\mathrm{proét}}}}
\newcommand{\affproet}[1]{\ensuremath{#1^{\mathrm{aff}}_{\mathrm{proét}}}}
\newcommand{\et}[1]{\ensuremath{#1_{\mathrm{et}}}}
\newcommand{\fpqc}[1]{\ensuremath{X_{\mathrm{fpqc}}}}
\newcommand{\shv}[1]{\ensuremath{\mathrm{Shv}(#1)}}
\newcommand{\spec}[1]{\ensuremath{\mathrm{Spec}(#1)}}
\newcommand{\gammacont}{\ensuremath{\Gamma_{\mathrm{cont}}}}
\newcommand{\colim}{\ensuremath{\operatorname{colim}}}
\newcommand{\pro}[1]{\ensuremath{\mathrm{Pro}(#1)}}
\newcommand{\sep}{\textnormal{sep}}
\newcommand{\Hom}{\ensuremath{\operatorname{Hom}}}

\newcommand{\N}{\mathbb{N}}
%\renewcommand{\C}{\mathbb{C}}
\newcommand{\R}{\mathbb{R}}
\newcommand{\Z}{\mathbb{Z}}

\newcommand{\m}{{\mathfrak{m}}}
\newcommand{\n}{{\mathfrak{n}}}
\newcommand{\p}{{\mathfrak{p}}}
\newcommand{\q}{{\mathfrak{q}}}
\renewcommand{\O}{{\mathcal{O}}}

\DeclareMathOperator\Spec{Spec}
\DeclareMathOperator\AffSch{AffSch}
\DeclareMathOperator\ProFin{ProFin}
\DeclareMathOperator\Hens{Hens}

\newcommand{\tildering}[3]{\ensuremath{#1_{\widetilde{(#2, #3)}}}}
\newcommand{\tildeideal}[2]{\ensuremath{\widetilde{(#1, #2)}}}

\newcommand{\red}{\color{red}}

\newcommand{\stacksproject}[1]{{\cite[\href{https://stacks.math.columbia.edu/tag/#1}{Tag #1}]{stacks-project}}}
