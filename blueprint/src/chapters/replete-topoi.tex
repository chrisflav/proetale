\chapter{Replete categories}

Let $\mathcal{C}$ be a category.

\begin{definition}[{\cite[Definition 3.1.1]{proetale}}]
    We say $\mathcal{C}$ is \emph{replete} if epimorphisms are stable under sequential limits, i.e.,
    if $(F_n)_{n \in \N}$ is an inverse system in $\mathcal{C}$ that admits a limit and such that
    $F_{n+1} \to F_n$ is surjective for all $n \in \N$, then $\mathrm{lim}\; F_n \to F_n$ is
    surjective for all $n \in \N$.
    \label{def:replete}
\end{definition}

\begin{proposition}
    Assume $\mathcal{C}$ has inverse limits and let $\mathcal{C}$ be replete.
    If $(F_n)_{n \in \N}$ is an inverse system in $\mathrm{Ab}(\mathcal{C})$ with
    epimorphic transition maps, then $\mathrm{lim} \; F_n \cong \mathrm{R}\; \mathrm{lim} \; F_n$.
    % TODO: make this more precise
    \label{prop:lim-exact-replete}
\end{proposition}

\begin{definition}[{\cite[Definition 3.2.1]{proetale}}]
    An object $F$ of $\mathcal{C}$ is called \emph{weakly contractible} if every epimorphism
    $G \to F$ has a section. We say $\mathcal{C}$ is \emph{locally weakly contractible},
    if each $X$ in $\mathcal{C}$ admits an epimorphism from the coproduct of $Y_i$,
    where $Y_i$ is coherent and weakly contractible.
    % TODO: explain coherent
    \label{def:lwc}
\end{definition}

\begin{proposition}
    If $\mathcal{C}$ is locally weakly contractible, it is replete.
    \uses{def:lwc, def:replete}
    \label{prop:lwc-replete}
\end{proposition}
