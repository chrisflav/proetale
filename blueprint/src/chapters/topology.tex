\chapter{The pro-étale topology}

\section{Weakly étale morphisms of schemes}

\begin{definition}[Weakly étale morphisms, {\cite[Definition 4.1.1]{proetale}}]
    A map $f \colon Y \to X$ is \emph{weakly étale} if $f$ is flat and
    $\Delta_f\colon Y \to Y \times_{X} Y$ is flat.

    \lean{AlgebraicGeometry.WeaklyEtale}
    \leanok
    \label{def:weakly-etale-morphism}
\end{definition}

\begin{lemma}[Local property of weakly étale morphisms]
    $f \colon Y \to X$ is weakly étale if and only if
    for every affine open $U$ of $Y$ and every affine open $V$ of $X$ with $f(U) \subset V$,
    the induced ring homomorphism $\Gamma(X, V) \to \Gamma(Y, U)$ is weakly étale.
    \uses{def:weakly-etale-algebra, def:weakly-etale-morphism}
    \label{lemma:weakly-etale-hasringhomprop}
\end{lemma}

\begin{lemma}[{\cite[Lemma 4.1.6]{proetale}}]
    Weakly étale is stable under composition and base change.
    \uses{def:weakly-etale-morphism}
    \lean{AlgebraicGeometry.WeaklyEtale.isStableUnderComposition, AlgebraicGeometry.WeaklyEtale.isStableUnderBaseChange}
    \leanok
    \label{lemma:weakly-etale-stable}
\end{lemma}

\begin{proof}
    This follows because flat is stable under composition and base change.
    \leanok
\end{proof}

\begin{definition}[Quasi-compact cover]
    A jointly surjective family $(f_i \colon Y_i \to X)_{i \in I}$ of morphisms of schemes is
    \emph{quasi-compact} if for
    every affine open $U$ of $X$ there exist quasi-compact opens $V_i$ in $Y_i$ such that
    $U = \bigcup_{i \in  I} f_i(V_i)$.
    \lean{AlgebraicGeometry.Scheme.Cover.QuasiCompact}
    \leanok
    \label{def:qc-cover}
\end{definition}

\begin{definition}
    Let $\mathcal{P}$ be a morphism property of schemes and $X$ be a scheme.
    The \emph{small $\mathcal{P}$-site of $X$} is the category of $X$-schemes with
    structure morphism satisfying $\mathcal{P}$ and with covers given by quasi-compact,
    jointly surjective families of morphisms satisfying $\mathcal{P}$.
\end{definition}

\begin{definition}[The fpqc site]
    Let $X$ be a scheme. The \emph{fqpc site of $X$}, denoted by $\fpqc{X}$, is the
    $\mathcal{P}$-site of $X$ with $\mathcal{P} = \text{flat}$.

    \label{def:fpqc-site}
\end{definition}

\begin{proposition}
    Let $F$ be a presheaf on the category of schemes. Then $F$ is a sheaf in the
    fpqc topology if and only if it is a sheaf in the Zariski topology and satisfies
    the sheaf property for $\{V \to U\}$ with $V$, $U$ affine
    and $V \to U$ faithfully flat.
\end{proposition}

\begin{theorem}
    The fpqc site is subcanonical, i.e., every representable presheaf is a sheaf.
    \label{thm:fpqc-subcanonical}
\end{theorem}

\begin{definition}[The pro-étale site, {\cite[Definition 4.1.1]{proetale}}]
    Let $X$ be a scheme. The \emph{pro-étale site of $X$}, denoted by $\proet{X}$, is the
    $\mathcal{P}$-site of $X$ with $\mathcal{P} = \text{weakly étale}$.

    \uses{def:qc-cover, def:weakly-etale-morphism}
    \label{def:proetale-site}
\end{definition}

\begin{remark}
    Despite the name proétale, Definition \ref{def:proetale-site} does not mention pro-étale morphisms.
    The name is justified (see \cite[Remark 4.1.3]{proetale}), but we don't formalise the
    justification (yet).
\end{remark}

The central property of the pro-étale site is the following:

\begin{proposition}
    The category of sheaves on $\proet{X}$ is locally weakly contractible.
    \uses{def:lwc, def:proetale-site}
    \label{prop:proet-lwc}
\end{proposition}

\begin{corollary}
    The category of sheaves on $\proet{X}$ is replete.
    \uses{prop:proet-lwc, prop:lwc-replete}
    \label{prop:proet-replete}
\end{corollary}
