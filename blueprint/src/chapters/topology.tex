\chapter{The pro-étale topology}

\section{Weakly étale morphisms of schemes}

\begin{definition}[Weakly étale morphisms, {\cite[Definition 4.1.1]{proetale}}]
    A map $f \colon Y \to X$ is \emph{weakly étale} if $f$ is flat and
    $\Delta_f\colon Y \to Y \times_{X} Y$ is flat.

    \lean{AlgebraicGeometry.WeaklyEtale}
    \leanok
    \label{def:weakly-etale-morphism}
\end{definition}

\begin{lemma}[Local property of weakly étale morphisms]
    $f \colon Y \to X$ is weakly étale if and only if
    for every affine open $U$ of $Y$ and every affine open $V$ of $X$ with $f(U) \subset V$,
    the induced ring homomorphism $\Gamma(X, V) \to \Gamma(Y, U)$ is weakly étale.
    \uses{def:weakly-etale-algebra, def:weakly-etale-morphism}
    \label{lemma:weakly-etale-hasringhomprop}
\end{lemma}

\begin{lemma}[{\cite[Lemma 4.1.6]{proetale}}]
    Weakly étale is stable under composition and base change.
    \uses{def:weakly-etale-morphism}
    \lean{AlgebraicGeometry.WeaklyEtale.isStableUnderComposition, AlgebraicGeometry.WeaklyEtale.isStableUnderBaseChange}
    \leanok
    \label{lemma:weakly-etale-stable}
\end{lemma}

\begin{proof}
    This follows because flat is stable under composition and base change.
    \leanok
\end{proof}
