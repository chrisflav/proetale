\chapter{The pro-étale topology}

\section{Weakly étale morphisms of schemes}

\begin{definition}[Weakly étale morphisms, {\cite[Definition 4.1.1]{proetale}}]
    A map $f \colon Y \to X$ is \emph{weakly étale} if $f$ is flat and
    $\Delta_f\colon Y \to Y \times_{X} Y$ is flat.

    \lean{AlgebraicGeometry.WeaklyEtale}
    \leanok
    \label{def:weakly-etale-morphism}
\end{definition}

\begin{lemma}[Local property of weakly étale morphisms]
    $f \colon Y \to X$ is weakly étale if and only if
    for every affine open $U$ of $Y$ and every affine open $V$ of $X$ with $f(U) \subset V$,
    the induced ring homomorphism $\Gamma(X, V) \to \Gamma(Y, U)$ is weakly étale.
    \uses{def:weakly-etale-algebra, def:weakly-etale-morphism}
    \label{lemma:weakly-etale-hasringhomprop}
\end{lemma}

\begin{lemma}
    Etale morphisms are weakly étale.
    \uses{def:weakly-etale-morphism}
    \label{lemma:etale-weakly-etale}
\end{lemma}

\begin{lemma}[{\cite[Lemma 4.1.6]{proetale}}]
    Weakly étale is stable under composition and base change.
    \uses{def:weakly-etale-morphism}
    \lean{AlgebraicGeometry.WeaklyEtale.isStableUnderComposition, AlgebraicGeometry.WeaklyEtale.isStableUnderBaseChange}
    \leanok
    \label{lemma:weakly-etale-stable}
\end{lemma}

\begin{proof}
    This follows because flat is stable under composition and base change.
    \leanok
\end{proof}

\section{The site}

\begin{definition}[Quasi-compact cover]
    A jointly surjective family $(f_i \colon Y_i \to X)_{i \in I}$ of morphisms of schemes is
    \emph{quasi-compact} if for
    every affine open $U$ of $X$ there exist quasi-compact opens $V_i$ in $Y_i$ such that
    $U = \bigcup_{i \in  I} f_i(V_i)$.
    \lean{AlgebraicGeometry.Scheme.Cover.QuasiCompact}
    \leanok
    \label{def:qc-cover}
\end{definition}

\begin{lemma}
    Let $(f_i \colon Y_i \to X)_{i \in I}$ be a jointly-surjective family of morphisms of schemes.
    If $f_i$ is flat and locally of finite presentation for every $i \in I$,
    the family is quasi-compact.
    \uses{def:qc-cover}
    \label{lemma:qc-cover-of-flat-fp}
\end{lemma}

\begin{proof}
    In this case, every $f_i$ is an open map, from which the claim easily follows.
\end{proof}

\begin{definition}
    Let $\mathcal{P}$ be a morphism property of schemes and $X$ be a scheme.
    The \emph{small $\mathcal{P}$-site of $X$} is the category of $X$-schemes with
    structure morphism satisfying $\mathcal{P}$ and with covers given by quasi-compact,
    jointly surjective families of morphisms satisfying $\mathcal{P}$.
    \label{def:small-P-site}
\end{definition}

\begin{definition}[The fpqc site]
    Let $X$ be a scheme. The \emph{fqpc site of $X$}, denoted by $\fpqc{X}$, is the
    $\mathcal{P}$-site of $X$ with $\mathcal{P} = \text{flat}$.
    \uses{def:small-P-site}
    \label{def:fpqc-site}
\end{definition}

\begin{proposition}
    Let $F$ be a presheaf on the category of schemes. Then $F$ is a sheaf in the
    fpqc topology if and only if it is a sheaf in the Zariski topology and satisfies
    the sheaf property for $\{V \to U\}$ with $V$, $U$ affine
    and $V \to U$ faithfully flat.
    \uses{def:fpqc-site}
    \label{prop:fpqc-sheaf-iff}
\end{proposition}

\begin{theorem}
    The fpqc site is subcanonical, i.e., every representable presheaf is a sheaf.
    \uses{def:fpqc-site}
    \label{thm:fpqc-subcanonical}
\end{theorem}

\begin{proof}
    TBA. \uses{prop:fpqc-sheaf-iff}
\end{proof}

\begin{definition}[The pro-étale site, {\cite[Definition 4.1.1]{proetale}}]
    Let $X$ be a scheme. The \emph{pro-étale site of $X$}, denoted by $\proet{X}$, is the
    $\mathcal{P}$-site of $X$ with $\mathcal{P} = \text{weakly étale}$.

    \uses{def:qc-cover, def:weakly-etale-morphism, def:small-P-site}
    \label{def:proetale-site}
\end{definition}

\begin{remark}
    Despite the name proétale, Definition \ref{def:proetale-site} does not mention pro-étale morphisms.
    The name is justified (see \cite[Remark 4.1.3]{proetale}), but we don't formalise the
    justification (yet).
\end{remark}

\begin{proposition}[{\cite[Lemma 4.2.4]{proetale}}]
    The site $\proet{X}$ is subcanonical.
    \label{prop:proet-subcanonical}
\end{proposition}

\begin{proof}
    Follows from \ref{thm:fpqc-subcanonical}, since every pro-étale sheaf is an fpqc-sheaf.
    \uses{thm:fpqc-subcanonical}
\end{proof}

\subsection{Affine covers}

\begin{definition}[{\cite[Definition 4.2.1]{proetale}}]
    An object $U$ of $\proet{X}$ is called a \emph{pro-étale affine} if it can be written
    as $U = \lim U_i$ over a small cofiltered diagram $i \mapsto U_i$ of affine schemes étale over $X$.
    The subcategory of pro-étale affines in $\proet{X}$ is denoted by $\affproet{X}$.
    \label{def:affproet}
\end{definition}

\begin{lemma}[{\cite[Remark 4.2.3]{proetale}}]
    The category $\affproet{X}$ admits limits indexed by a connected diagram and the
    forgetful functor to $\mathrm{Sch} / X$ preserves them.
    If $X$ is affine, $\affproet{X}$ has all small limits.
    \uses{def:affproet}
    \label{lemma:affproet-limits}
\end{lemma}

\begin{definition}
    Let $X$ be affine. The \emph{affine pro-étale site} of $X$ is the category $\affproet{X}$ with
    the covering families are
    the quasi-compact covers.
    \uses{lemma:affproet-limits, def:affproet}
    \label{def:affproet-site}
\end{definition}

\begin{lemma}[{\cite[Lemma 4.2.4]{proetale}}]
    The topos $\shv{\proet{X}}$ is generated by $\affproet{X}$.
    % TODO: make this precise
    \uses{def:affproet}
    \label{lemma:affproet-gen-proet}
\end{lemma}

\subsection{Repleteness}

An important property of the pro-étale site is the following:

\begin{proposition}[{\cite[Proposition 4.2.8]{proetale}}]
    The category of sheaves on $\proet{X}$ has enough weakly contractible objects.
    \uses{def:weakly-contractible, def:proetale-site}
    \label{prop:proet-wc}
\end{proposition}

\begin{corollary}
    The category of sheaves on $\proet{X}$ is replete.
    \uses{prop:proet-wc, prop:wc-replete}
    \label{prop:proet-replete}
\end{corollary}

\section{Comparison with étale cohomology}

Let $X$ be a scheme. Since an étale map is also weakly étale by \ref{lemma:etale-weakly-etale} we may
define:

\begin{definition}
    We denote by $\nu = \nu_X$ the morphism of sites $\proet{X} \to \et{X}$ induced by the functor
    from $\et{X}$ to $\proet{X}$ given by $(U \to X) \mapsto (U \to X)$.
    \uses{lemma:etale-weakly-etale, def:proetale-site}
    \label{def:forget-proet}
\end{definition}

\begin{lemma}
    The morphism of sites $\nu$ is continuous.
    \uses{def:forget-proet}
    \label{lemma:forget-proet-continuouos}
\end{lemma}

\begin{proof}
    This follows from the definitions, because every étale covering is
    a quasi-compact covering by \ref{lemma:qc-cover-of-flat-fp}.
    \uses{lemma:qc-cover-of-flat-fp}
\end{proof}

We denote by $\nu_{*} \colon \shv{\proet{X}} \to \shv{\et{X}}$ and
$\nu^* \colon \shv{\et{X}} \to \shv{\proet{X}}$ the corresponding pushforward and pullback functors.
Since $\nu^*$ is exact, we have $R \nu^* = \nu^*$ by abuse of notation and we simply write $\nu^*$ everywhere.
In contrast to \cite{proetale}, we write $R \nu_{*}$ for the derived functor of $\nu_{*}$.

\begin{lemma}[{\cite[Lemma 5.1.1]{proetale}}]
    Let $F$ be a sheaf on $\et{X}$ and $U = \lim_i U_i$ in $\affproet{X}$. Then
    $\nu^*F(U) = \colim_i F(U_i)$.
    \uses{lemma:forget-proet-continuous, def:affproet}
    \label{lemma:pullback-section-affproet}
\end{lemma}

\begin{lemma}[{\cite[Lemma 5.1.6]{proetale}}]
    Let $K$ be in $D^+(\et{X})$ and $U = \lim_i U_i$ in $\affproet{X}$. Then
    $R \Gamma(U, \nu^*K) = \colim_i R\Gamma(U_i, K)$.
    \uses{lemma:forget-proet-continuous}
    \label{lemma:derived-pullback-section-affproet}
\end{lemma}

\begin{proof}
    TBA. \uses{lemma:pullback-section-affproet}
\end{proof}

\begin{proposition}[{\cite[Corollary 5.1.6]{proetale}}]
    For any $K$ in $D^{+}(\et{X})$ the counit $K \to R \nu_{*} \nu^* K$ is an isomorphism.
    \label{prop:proetale-etale-iso}
    \uses{lemma:forget-proet-continuous}
\end{proposition}

\begin{proof}
    TBA. \uses{lemma:derived-pullback-section-affproet, lemma:affproet-gen-proet}
\end{proof}

\begin{corollary}
    Let $K$ be in $D^{+}(\et{X})$. Then
    \[
        R \Gamma(\et{X}, K) \cong R \Gamma(\proet{X}, \nu^{*} K)
    .\]
    \uses{lemma:forget-proet-continuous}
    \label{cor:derived-sections-proetale-etale-iso}
\end{corollary}

\begin{proof}
    By \ref{prop:proetale-etale-iso}, we have
    $R \Gamma(\et{X}, K) \cong R \Gamma(\et{X}, R \nu_{*}\nu^*K)$. Since
    $\nu_*$ maps injective sheaves to acyclic ones by general theory, we have
    $R \Gamma(\et{X}, -) \circ \nu_{*} = R (\Gamma(\et{X}, -) \circ \nu_*) = R \Gamma(\proet{X}, -)$.
    So the claim follows.
    \uses{prop:proetale-etale-iso}
\end{proof}

\begin{definition}[{\cite[\S3]{jannsen}}]
    Denote by $\gammacont(X, -)\colon \mathrm{Ab}(\et{X})^{\N} \to \mathrm{Ab}$ the functor
    $$\left( F_n \right)_{n \in \N} \mapsto \Gamma\left(X, \lim F_n\right).$$ For an inverse
    system $(F_n)_{n \in \N}$, we call the cohomology groups
    $H^i \mathrm{R} \gammacont\left(X, -\right)\left( \left( F_{n} \right)_{n \in \N} \right) $,
    denoted by $H^i_{\mathrm{cont}}(\et{X}, (F_{n})_{n \in \N})$,
    the \emph{continuous étale cohomology} of $X$ with coefficients in $\left( F_n \right)_{n \in \N}$.

    \label{def:continuous-etale-cohomology}
\end{definition}

\begin{theorem}[{\cite[Proposition 5.6.2]{proetale}}]
    Let $(F_n)_{n \in \N}$ be an inverse system of abelian sheaves on $\et{X}$ with
    epimorphic transition maps. Then there exists a canonical identification
    \[
        R \Gamma_{cont}(X, (F_n)_n) \cong R \Gamma(\proet{X}, \lim \nu^* F_n)
    .\] In particular, for every $i \ge 0$ we have
    \[
        H^i_{\mathrm{cont}}(\et{X}, (F_n)_n) \cong H^i(\proet{X}, \lim \nu^* F_n)
    .\]
    \uses{def:continuous-etale-cohomology, lemma:forget-proet-continuous}
    \label{thm:comparison-continuous}
\end{theorem}

\begin{proof}
    We repeatedly use the fact that $\Gamma$ commutes with inverse limits both on $\et{X}$ and $\proet{X}$.
    Then we have
    \begin{align*}
        R \gammacont(X, (F_n)_n) &\cong R \left( \Gamma(\et{X}, -) \circ \lim \right) ( (F_n)_n ) \\
                                    &\cong R \left( \lim \circ \Gamma(\et{X}, -) \right) ( (F_n)_n ) \\
                                    &\cong R \lim ( R \Gamma(\et{X}, F_n)) \\
                                    &\cong R \lim ( R \Gamma(\proet{X}, \nu^{*} F_n)) \\
                                    &\cong R \left( \lim \circ \Gamma(\proet{X}, -) \right) ( (\nu^* F_n)_n ) \\
                                    &\cong R \left( \Gamma(\proet{X}, -) \circ \lim \right) ( (\nu^* F_n)_n ) \\
                                    &\cong R \Gamma(\proet{X}, R \lim \nu^* F_n) \\
                                    &\cong R \Gamma(\proet{X}, \lim \nu^* F_n) \\
    \end{align*}
    The fourth isomorphism is \ref{cor:derived-sections-proetale-etale-iso} and the last
    is \ref{prop:lim-exact-replete}.
    \uses{prop:proet-replete, prop:lim-exact-replete, cor:derived-sections-proetale-etale-iso}
\end{proof}
