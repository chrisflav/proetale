\chapter{Local structure}

This chapter serves for the commutative algebra preparation of the remaining part of the paper. The key results are
\begin{enumerate}
  \item \Cref{thm:ind-etale-w-contractible-cover} expressing that every affine scheme has a w-contractible cover in the proetale site.
  \item \Cref{thm:weakly-etale-ind-etale} expressing that every weakly etale map can be covered by ind-etale maps.
\end{enumerate}

% Question: Can we avoid using weakly etale algebra, only use ind-etale algebra? This would need a modification of theorem 5.4.2, by directly mimicking the proof of topological invariance of étale sites for weakly étale.
% If so, we can remove theorem \Cref{thm:weakly-etale-ind-etale} and every dependency of it. 
\section{Preliminaries}

\begin{definition}[Extremally disconnected space]
  \label{def:extremally-disconnected}
  \lean{ExtremallyDisconnected}
  \mathlibok

  A topological space \(X\) is \emph{extremally disconnected} if the closure of every open subset is open.
\end{definition}

\begin{definition} [Stone-Čech compactification]
  \label{def:stone-cech-compactification}
  \lean{StoneCech}
  \mathlibok

  Let \(X\) be a topological space. The \emph{Stone-Čech compactification} of \(X\) is the
  profinite space \(\beta(X)\) such that
  \begin{enumerate}
    \item \(X\) is dense in \(\beta(X)\);
    \item every continuous map \(f: X \to Y\) to a compact Hausdorff space \(Y\)
        extends uniquely to a continuous map \(\beta(f): \beta(X) \to Y\).
  \end{enumerate}
  We denote the Stone-Čech compactification of \(X\) by \(\beta(X)\).
\end{definition}

\begin{theorem}
    Let \(X\) be a topological space. Then the Stone-Čech compactification \(\beta(X)\) is extremally disconnected.
    \label{thm:stone-cech-extremally-disconnected}
    \uses{def:extremally-disconnected, def:stone-cech-compactification}
    \lean{StoneCech.projective}
    \mathlibok
\end{theorem}

\begin{proposition}[{\cite[\href{https://stacks.math.columbia.edu/tag/090D}{Tag 090D}]{stacks-project}}]
  Let $X$ be a quasi-compact Hausdorff space. There exists a continuous surjection $X' \to X$ with $X'$ quasi-compact, Hausdorff, and extremally disconnected.
  \label{thm:extremally-disconnected-cover}
  \uses{def:extremally-disconnected, def:stone-cech-compactification}
\end{proposition}

\begin{proof}
  \uses{thm:stone-cech-extremally-disconnected}
  Let $Y=X$ but endowed with the discrete topology. Let $X'=\beta (Y)$, which is extremally disconnected by \Cref{thm:stone-cech-extremally-disconnected}. The continuous map $Y \to X$ factors as $Y \to X' \to X$. 
\end{proof}

% Move this to a better place.
\begin{definition}[Local isomorphisms, {\cite[\href{https://stacks.math.columbia.edu/tag/096E}{Tag 096E}]{stacks-project}}]
  \label{def:local-isomorphism}

  We say $A \to B$ is a \emph{local isomorphism} if for every prime $\mathfrak{q} \subset B$ there exists a $g \in B$, $g \notin \mathfrak{q}$ such that $A \to B_g$ induces an open immersion $\Spec (B_g) \to \Spec (A)$.
\end{definition}

\section{Ind-etale and weakly etale ring maps}

\begin{definition}[Ind-étale algebras, {\cite[\href{https://stacks.math.columbia.edu/tag/097I}{Tag 097I}]{stacks-project}}]
    An $R$-algebra $S$ is \emph{ind-étale} if it is a filtered colimit of étale $R$-algebras.
    We say a ring homomorphism $R \to S$ is \emph{ind-étale} if $S$ is ind-étale as an $R$-algebra via $f$.
    \label{def:ind-etale}
\end{definition}

\section{w-local spaces}

\begin{definition}[w-local spaces, {\cite[Definition 2.1.1]{proetale}}, {\cite[\href{https://stacks.math.columbia.edu/tag/096A}{Tag 096A}]{stacks-project}}]
    \label{def:w-local-space}
    \lean{WLocalSpace}
    \leanok
    % Stacks uses another definition, split is a property. Every point of X specializes to a unique closed point. This is better.
    % To be modified.
    A topological space \(X\) is \emph{w-local} if it satisfies:
    \begin{enumerate}
        \item \(X\) is spectral;
        \item All open covers split, i.e., for every open cover \(\{U_i\}_{i \in I}\) of \(X\), the map \(\coprod_{i \in I} U_i \to X\) has a section;
        \item The subspace \(X^c\) of closed points is closed.
    \end{enumerate}
\end{definition}

\begin{definition}[w-local morphisms, {\cite[Definition 2.1.1]{proetale}}, {\cite[\href{https://stacks.math.columbia.edu/tag/096A}{Tag 096A}]{stacks-project}}]
    Let \(X\) and \(Y\) be w-local spaces. A morphism \(f: X \to Y\) is \emph{w-local} if it is spectral and the image of closed points \(f(X^c) \subseteq Y^c\).
    \label{def:w-local-space-map}
    \uses{def:w-local-space}
    \lean{IsWLocalMap}
    \leanok
\end{definition}

\begin{lemma}[{\cite[\href{https://stacks.math.columbia.edu/tag/096C}{Tag 096C}]{stacks-project}}]
Let $X$ be a spectral space. Let
\[
\begin{tikzcd}
Y \arrow[r] \arrow[d] & T \arrow[d] \\
X \arrow[r] & \pi_0(X)
\end{tikzcd}
\]
be a cartesian diagram in the category of topological spaces with $T$ profinite. Then $Y$ is spectral and $T = \pi_0(Y)$. If moreover $X$ is w-local, then $Y$ is w-local, $Y \to X$ is w-local, and the set of closed points of $Y$ is the inverse image of the set of closed points of $X$.
\label{thm:cartesian-w-local}
\uses{def:w-local-space,def:w-local-space-map}
\end{lemma}

\begin{proof}
  TBA.
\end{proof}

\section{w-local rings}

\begin{definition}[w-local rings, {\cite[Definition 2.2.1(i)]{proetale}}]
    A ring \(A\) is \emph{w-local} if Spec(A) is w-local.
    \label{def:w-local-ring}
    \uses{def:w-local-space}
    \lean{WLocalRing}
    \leanok
\end{definition}

\begin{definition}[w-local ring maps, {\cite[Definition 2.2.1(iii)]{proetale}}]
    A ring map \(f: A \to B\) between w-local rings is \emph{w-local} if the induced map of w-local spaces \(\Spec(f) : \Spec(B) \to \Spec(A)\) is w-local.
    % Spectral is automatic, only sending closed pts to closed pts matters.

    \label{def:w-local-ring-map}
    \uses{def:w-local-ring, def:w-local-space, def:w-local-space-map}
    %\lean{IsWLocalRingMap}
    %\leanok
\end{definition}


\begin{lemma}[{\cite[\href{https://stacks.math.columbia.edu/tag/097A}{Tag 097A}]{stacks-project}}]
\label{thm:closed-points-isom-w-local}
\uses{def:w-local,def:w-local-ring-map,def:ind-Zariski}

Let $A$ be a w-local ring. Let $I \subset A$ be the radical ideal cutting out the set $X^c$ of closed points in $X = \Spec (A)$. Let $A \to B$ be a ring map inducing algebraic extensions on residue fields at primes. Then
\begin{enumerate}
    \item every point of $Z = V(IB)$ is a closed point of $\Spec (B)$,
    \item there exists an ind-Zariski ring map $B \to C$ such that
    \begin{enumerate}
        \item $B/IB \to C/IC$ is an isomorphism,
        \item $C$ is w-local,
        \item the induced map $p: Y \to X$ is w-local, and
        \item $p^{-1}(X^c)$ is the set of closed points of $Y = \Spec (C)$, where $p : Y \to X$ is the induced map of $A \to B \to C$.
    \end{enumerate}
\end{enumerate}
\end{lemma}

\begin{proof}
  TBA.
\end{proof}



\begin{definition}[w-strictly local rings, {\cite[Definition 2.2.1(ii)]{proetale}}]
    A ring \(A\) is \emph{w-strictly local} if \(A\) is w-local, and every local ring of \(A\) at a maximal ideal is strictly henselian.
    % TODO: every faithfully flat \'etale map \(A \to B\) has a retraction. this should be a property.
    \label{def:w-strictly-local-ring}
    \uses{def:w-local-ring}
    % refactor WStrictlyLocalRing, using the definition of strict henselian.
    % \lean{WStrictlyLocalRing}
    % \leanok
\end{definition}

\begin{lemma}[{\cite[\href{https://stacks.math.columbia.edu/tag/097X}{Tag 097X}]{stacks-project}}]
  For any ring \(A\), there exists an ind-étale faithfully flat \(A\)-algebra \(C\) with \(C\) w-strictly local.
  \label{thm:ind-etale-w-strictly-local-cover}
  \uses{def:w-strictly-local-ring,def:ind-etale}
\end{lemma}

\begin{proof}
  TBA
\end{proof}

\section{ind-Zariski maps}

\begin{definition}[Ind-(Zariski localizations), {\cite[\href{https://stacks.math.columbia.edu/tag/096N}{Tag 096N}]{stacks-project}}]
    \label{def:ind-zariski}

    An $R$-algebra $S$ is called a \emph{ind-Zariski} if $S$ can be written as a filtered colimit $S \simeq \colim S_i$ with each $R \to S_i$ a local isomorphism. A ring homomorphism $f : R \to S$ is called a \emph{ind-Zariski localization} if $S$ is ind-Zariski localization as an $R$-algebra via $f$.

    \uses{def:local-isomorphism}
\end{definition}

\begin{remark}
  The definition of ind-Zariski is slightly more general from \cite[Definition 2.2.1(iv)]{proetale}, where they defined as a filtered colimit of finite products of principal localizations. 
\end{remark}

\begin{lemma}[{\cite[\href{https://stacks.math.columbia.edu/tag/096T}{Tag 096T}]{stacks-project}}]
  Let $A \to B$ be an ind-Zariski ring map. Then it identifies local rings, i.e. for every prime $\mathfrak{q} \subset B$ the canonical map $A_{\varphi^{-1}(\mathfrak{q})} \to B_{\mathfrak{q}}$ is an isomorphism.
  \label{ind-Zariski-identifies-local-rings}
  \uses{def:ind-zariski}
\end{lemma}

\begin{proof}
  Omitted.
\end{proof}

\begin{lemma}[{\cite[\href{https://stacks.math.columbia.edu/tag/097D}{Tag 097D}]{stacks-project}}]
Let $A$ be a ring and let $X = \Spec (A)$. Let $T$ be a profinite space and let $T \to \pi_0(X)$ be a continuous map. There exists an ind-Zariski ring map $A \to B$ such that with $Y = \Spec (B)$ the diagram
\[
\begin{tikzcd}
Y \arrow[r] \arrow[d] & \pi_0(Y) \arrow[d] \\
X \arrow[r] & \pi_0(X)
\end{tikzcd}
\]
is cartesian in the category of topological spaces and such that $\pi_0(Y) = T$ as spaces over $\pi_0(X)$.
\label{exists-ind-zariski-cartesian}
\uses{def:ind-zariski}
\end{lemma}

\begin{proof}
  TBA
\end{proof}

\section{w-contractible rings}

\begin{definition}[w-contractible rings, {\cite[Definition 2.4.1]{proetale}}]
A ring \(A\) is \emph{w-contractible} if every faithfully flat ind-\'etale map \(A \to B\) has a retraction.
  \label{def:w-contractible-ring}
\end{definition}

\begin{lemma}[{\cite[\href{https://stacks.math.columbia.edu/tag/0982}{Tag 0982}]{stacks-project}}]
\label{thm:w-contractible-iff}
\uses{def:w-contractible-ring,def:w-strictly-local-ring,def:extremally-disconnected}

Let $A$ be a ring. The following are equivalent:
\begin{enumerate}
    \item $A$ is w-contractible, and
    \item $A$ satisfies
    \begin{enumerate}
        \item[a] $A$ is w-strictly local, and
        \item[b] $\pi_0(\Spec (A))$ is extremally disconnected.
    \end{enumerate}
\end{enumerate}
\end{lemma}

\begin{proof}
  \uses{thm:closed-points-isom-w-local}

  Assume 2a and 2b. Let $A \to B$ be faithfully flat and ind-\'etale. 
  % We will use without further mention the fact that a flat map $A \to B$ is faithfully flat if and only if every closed point of $\operatorname{Spec}(A)$ is in the image of $\operatorname{Spec}(B) \to \operatorname{Spec}(A)$. 
  We will show that $A \to B$ has a retraction.

  Let $I \subset A$ be an ideal such that $V(I) \subset \Spec (A)$ is the set of closed points of $\Spec (A)$. We may replace $B$ by the ring $C$ constructed in \Cref{thm:closed-points-isom-w-local} for $A \to B$ and $I \subset A$. Thus we may assume $\Spec (B)$ is w-local such that the set of closed points of $\Spec (B)$ is $V(IB)$. 
  % In this case $A \to B$ identifies local rings by condition (3)(c) as it suffices to check this at maximal ideals of $B$ which lie over maximal ideals of $A$. % This uses ind-etale/field is algebraic extension. ind-etale is preserved under base change and ind-Zariski is ind-etale, composition of ind-etale is ind-etale.
  % Thus $A \to B$ has a retraction by Lemma \ref{lemma:09AZ}.

  % Assume (1) or equivalently (2). We have (3)(c) by Lemma \ref{lemma:097V}. Properties (3)(a) and (3)(b) follow from Lemma \ref{lemma:09AZ}.
\end{proof}
  

\begin{lemma}
  If a ring \(C\) is w-strictly local, then there exists an ind-étale faithfully flat \(C\)-algebra \(D\) with \(D\) w-contractible.
  \label{thm:ind-etale-w-contractible-cover-of-w-strictly-local}
  \uses{def:w-contractible-ring,def:w-strictly-local-ring,def:ind-etale}
\end{lemma}

\begin{proof}
  % The proof goes as in https://stacks.math.columbia.edu/tag/0983, starting from the second sentence.
  \uses{thm:extremally-disconnected-cover,exists-ind-zariski-cartesian,thm:cartesian-w-local,ind-Zariski-identifies-local-rings,thm:w-contractible-iff}

  Choose an extremally disconnected space $T$ and a surjective continuous map $T \to \pi_0(\operatorname{Spec}(C))$ by \Cref{thm:extremally-disconnected-cover}. Note that $T$ is profinite.
  Apply \Cref{exists-ind-zariski-cartesian} to find an ind-Zariski ring map $C \to D$ such that $\pi_0(\Spec (D)) \to \pi_0(\Spec (C))$ realizes $T \to \pi_0(\Spec (C))$ and such that
  \[
  \begin{tikzcd}
  \operatorname{Spec}(D) \arrow[r] \arrow[d] & \pi_0(\operatorname{Spec}(D)) \arrow[d] \\
  \operatorname{Spec}(C) \arrow[r] & \pi_0(\operatorname{Spec}(C))
  \end{tikzcd}
  \]
  is cartesian in the category of topological spaces.

  Following \Cref{thm:cartesian-w-local}, we note that $\Spec (D)$ is w-local, that $\Spec (D) \to \Spec (C)$ is w-local, and that the set of closed points of $\Spec (D)$ is the inverse image of the set of closed points of $\Spec (C)$.

  Thus it is still true that the local rings of $D$ at its maximal ideals are strictly henselian (as they are isomorphic to the local rings at the corresponding maximal ideals of $C$ by \Cref{ind-Zariski-identifies-local-rings}). It follows from \Cref{thm:w-contractible-iff} that $D$ is w-contractible.
\end{proof}

\begin{theorem}[{\cite[Theorem 2.4.9]{proetale}}, {\cite[\href{https://stacks.math.columbia.edu/tag/0983}{Tag 0983}]{stacks-project}}]
  For any ring $A$, there exists an ind-étale faithfully flat $A$-algebra $D$ with $D$ w-contractible.

  \label{thm:ind-etale-w-contractible-cover-exists}
  \uses{def:w-contractible-ring,def:ind-etale}
\end{theorem}

\begin{proof}
  \uses{thm:ind-etale-w-strictly-local-cover,thm:ind-etale-w-contractible-cover-of-w-strictly-local}
  This is a combination of \Cref{thm:ind-etale-w-strictly-local-cover} and \Cref{thm:ind-etale-w-contractible-cover-of-w-strictly-local}.
\end{proof}
