\chapter{Local structure}

% \section{Preliminaries}

% \begin{definition} [Extremally disconnected space]
%     A topological space \(X\) is \emph{extremally disconnected} if the closure of every open subset is open.
%     \label{def:extremally-disconnected}
%     \lean{ExtremallyDisconnected}
%     \leanok
% \end{definition}

% \begin{definition} [Stone-Čech compactification]
%     Let \(X\) be a topological space. The \emph{Stone-Čech compactification} of \(X\) is the
%     profinite space \(\beta(X)\) such that
%     \begin{enumerate}
%         \item \(X\) is dense in \(\beta(X)\);
%         \item every continuous map \(f: X \to Y\) to a compact Hausdorff space \(Y\)
%             extends uniquely to a continuous map \(\beta(f): \beta(X) \to Y\).
%     \end{enumerate}
%     We denote the Stone-Čech compactification of \(X\) by \(\beta(X)\).
%     \label{def:stone-cech-compactification}
%     \lean{StoneCech}
%     \leanok
% \end{definition}

\begin{theorem}
    Let \(X\) be a topological space. Then the Stone-Čech compactification \(\beta(X)\) is extremally disconnected.
    \label{thm:stone-cech-extremally-disconnected}
    % \uses{def:extremally-disconnected, def:stone-cech-compactification}
    \lean{StoneCech.projective}
\end{theorem}

% \begin{proof}
%     Omitted.
%     \leanok
% \end{proof}


\section{Weakly étale algebras}

\begin{definition}[Weakly étale algebras, {\cite[Definition 2.3.1]{proetale}}]
    An $R$-algebra $S$ is \emph{weakly étale} if $S$ is flat over $R$ and
    $S$ is flat over $S \otimes_{R} S$. We say a ring homomorphism
    $R \to S$ is \emph{weakly étale} if $S$ is weakly étale as an $R$-algebra via $f$.
    \label{def:weakly-etale-algebra}
\end{definition}

\begin{definition}[Ind-étale algebras, {\cite[\href{https://stacks.math.columbia.edu/tag/097I}{Tag 097I}]{stacks-project}}]
    An $R$-algebra $S$ is \emph{ind-étale} if it is a filtered colimit of étale $R$-algebras.
    We say a ring homomorphism $R \to S$ is \emph{ind-étale} if $S$ is ind-étale as an $R$-algebra via $f$.
    \label{def:ind-etale-algebra}
\end{definition}

The following relation between weakly étale algebra and ind-étale algebra is the main result of this section.
\begin{theorem}[{\cite[Theorem 2.3.4]{proetale}}]
Let $f: A \to B$ be weakly étale. Then there exists a faithfully flat ind-étale morphism $g: B \to C$ such that $g \circ f: A \to C$ is ind-étale.
    \label{thm:weakly-etale-ind-etale}
\end{theorem}

Check whether theorem 5.4.2 can be refactored to avoid use the above theorem by directly mimicking the proof of topological invariance of étale sites for weakly étale.

Remove 5.4.1



The local version of Theorem 2.3.4 follows from the following result of Olivier, \cite{Oli72}:

\begin{theorem}[{\cite[Theorem 2.3.5]{proetale}}]
Let $A$ be a strictly henselian local ring, and let $B$ be a weakly étale local $A$-algebra. Then $f: A \to B$ is an isomorphism.
    \label{thm:weakly-etale-local}
\end{theorem}


\section{w-local spaces}

Goal: For every spectral space \(X\), construct the pro-(open cover) \(X^Z \to X\) that admits no further non-split open covers.

\begin{definition}[w-local spaces, {\cite[Definition 2.1.1]{proetale}}]
    \label{def:w-local-space}
    A topological space \(X\) is \emph{w-local} if it satisfies:
    \begin{enumerate}
        \item \(X\) is spectral;
        \item All open covers split, i.e., for every open cover \(\{U_i\}_{i \in I}\) of \(X\), the map \(\coprod_{i \in I} U_i \to X\) has a section;
        \item The subspace \(X^c\) of closed points is closed.
    \end{enumerate}
\end{definition}

\begin{definition}[w-local morphisms, {\cite[Definition 2.1.1]{proetale}}]
    Let \(X\) and \(Y\) be w-local spaces. A morphism \(f: X \to Y\) is \emph{w-local} if it is spectral and the image of closed points \(f(X^c) \subseteq Y^c\).
    \label{def:w-local-morphism}
    \uses{def:w-local-space}
\end{definition}

\begin{definition}
    The category of w-local spaces is denoted by \(\S^{wl}\). The objects are w-local spaces, and the morphisms are w-local morphisms.
    \label{def:w-local-space-category}
    \uses{def:w-local-space, def:w-local-morphism}
\end{definition}

% TBA: S: category of spectral spaces, i : \(\S^{wl}\) is the full subcategory of \(\S\) consisting of w-local spaces.


\section{w-local rings}

\begin{definition}[w-local rings, {\cite[Definition 2.2.1(i)]{proetale}}]
    A ring \(A\) is \emph{w-local} if Spec(A) is w-local.
    \label{def:w-local-ring}
    \uses{def:w-local-space}
\end{definition}

\begin{definition}[w-strictly local rings, {\cite[Definition 2.2.1(ii)]{proetale}}]
    A ring \(A\) is \emph{w-strictly local} if \(A\) is w-local, and every faithfully flat \'etale map \(A \to B\) has a section.
    \label{def:w-strictly-local-ring}
    \uses{def:w-local-ring}
\end{definition}

\begin{definition}[w-local ring maps, {\cite[Definition 2.2.1(iii)]{proetale}}]
    A ring map \(f: A \to B\) between w-local rings is \emph{w-local} if the induced map of w-local spaces \(\Spec(f) : \Spec(B) \to \Spec(A)\) is w-local.
    \label{def:w-local-ring-map}
    \uses{def:w-local-ring, def:w-local-space, def:w-local-morphism}
\end{definition}

\section{w-contractible rings}

\begin{definition}[w-contractible rings, {\cite[Definition 2.4.1]{proetale}}]
A ring \(A\) is \emph{w-contractible} if every faithfully flat ind-\'etale map \(A \to B\) has a section.
    \label{def:w-contractible-ring}
\end{definition}

The last theorem in this section used before \cite[Theorem 5.6.2]{proetale} is the following \cref{thm:ind-etale-w-contractible}

\begin{lemma}[{\cite[Theorem 2.4.9]{proetale}}]
For any ring $A$, there is an ind-\'etale faithfully flat $A$-algebra $A_0$ with $A_0$ w-contractible.
    \label{thm:ind-etale-w-contractible-cover-exists}
    \uses{def:w-contractible-ring}
\end{lemma}

\begin{proof}
    \uses{thm:stone-cech-extremally-disconnected}

\begin{enumerate}
\item Choose an ind-\'etale faithfully flat $A^{Z}/I_{A^{Z}}$-algebra $A'_0$ with $A'_0$ w-strictly local and $\Spec(A'_0)$ an extremally disconnected profinite set; this is possible by
    \begin{enumerate}
    \item Since $A^{Z}/I_{A^{Z}}$ is absolutely flat (Lemma 2.2.3), we can find an ind-\'etale faithfully flat $A^{Z}/I_{A^{Z}}$-algebra \(\bar{A}\) by Lemma 2.2.7
    \item Choose a extremally disconnected profinite set $Y$ covers \(X = (\Spec \bar{A})^c\) (We can choose \(Y\) to be \(\beta(X)\), the Stone-\v{C}ech compactification of \(X\));
    \item By the functor \(S \mapsto S\times_{\pi_0(X)} X\) from \(\ProFin_{/\pi_0(X)} \to \Aff_{/X}\) in Lemma 2.2.8, we get a ind-(Zariski localization) ring map \(f: \bar{A} \to A'_0 \), such that \((\Spec A'_0)^c = S\). The construction \(A'_0\) and the map \(f\) is w-local as mentioned in Lemma 2.1.14 (by Lemma 2.1.9 limit of w-local (along w-local map) is w-local). By the construction, the local ring of \(A'_0\) at every closed point is isomorphic to the local ring of its image in \(\bar{A}\), thus this \(f\) preserves w-strict locality by Lem 2.2.9(check closed point). {\color{red} State this as a lemma: Construction \(f\) preserves w-locality and w-strict locality.}
    \item As ind-(Zariski localization) is ind-\'etale faithfully flat (A new lemma here), so the composition ring map $A^{Z}/I_{A^{Z}}$ is ind-\'etale faithfully flat.
    \end{enumerate}

\item Let $A_0 = \operatorname{Hens}_{A^{Z}}(A'_0)$. 
    \begin{enumerate}
    \item By Lemma 2.4.8, $A'_0$ is w-contractible.
    \item Since \(\operatorname{Hens}_{A^{Z}}(-) \otimes A^{Z}/I_{A^{Z}} \simeq Id \) (Lemma 2.2.12), we have \(A_0 \otimes_{A^{Z}/I_{A^{Z}}} A^{Z} \simeq A'_0\).
    {\color{red} One more lemma show that \(A_0\) is Henselian along \(I_{A^{Z}} A_0\), why is this not in the paper?} Thus by Lemma 2.4.3, $A_0$ is w-contractible if and only if $ A_0 / I_{A^{Z}} A_0 \simeq A'_0 $ is. Indeed, $A_0$ is w-contractible.
    \item The map $A \to A_0$ is faithfully flat and ind-\'etale since both $A \to A^{Z}$ (Lemma 2.2.4) and $A^{Z} \to A_0$ ({\color{red} flat since colim of etale maps, faithfully flat since \(I_{A^{Z}} \) is the Jacobson ideal, maximal points unchanged when quotient by it. State this as a lemma.}) are so individually.
    \end{enumerate}
\end{enumerate}


\end{proof}

