\chapter{Local structure}

\section{Weakly étale algebras}

\begin{definition}[Weakly étale algebras, {\cite[Definition 2.3.1]{proetale}}]
    An $R$-algebra $S$ is \emph{weakly étale} if $S$ is flat over $R$ and
    $S$ is flat over $S \otimes_{R} S$. We say a ring homomorphism
    $R \to S$ is \emph{weakly étale} if $S$ is weakly étale as an $R$-algebra via $f$.
    \label{def:weakly-etale-algebra}
\end{definition}

\begin{definition}[Ind-étale algebras, {\cite[\href{https://stacks.math.columbia.edu/tag/097I}{Tag 097I}]{stacks-project}}]
    An $R$-algebra $S$ is \emph{ind-étale} if it is a filtered colimit of étale $R$-algebras.
    We say a ring homomorphism $R \to S$ is \emph{ind-étale} if $S$ is ind-étale as an $R$-algebra via $f$.
    \label{def:ind-etale-algebra}
\end{definition}

The following relation between weakly étale algebra and ind-étale algebra is the main result of this section.
\begin{theorem}[{\cite[Theorem 2.3.4]{proetale}}]
Let $f: A \to B$ be weakly étale. Then there exists a faithfully flat ind-étale morphism $g: B \to C$ such that $g \circ f: A \to C$ is ind-étale.
    \label{thm:weakly-etale-ind-etale}
\end{theorem}

The local version of Theorem 2.3.4 follows from the following result of Olivier, \cite{Oli72}:

\begin{theorem}[{\cite[Theorem 2.3.5]{proetale}}]
Let $A$ be a strictly henselian local ring, and let $B$ be a weakly étale local $A$-algebra. Then $f: A \to B$ is an isomorphism.
    \label{thm:weakly-etale-local}
\end{theorem}


\section{w-local spaces}

\section{w-local rings}

\section{w-contractible rings}

The last theorem in this section used before proposition 5.6.2 is theorem 2.4.9

